%!TEX root = ./main.tex

\chapter{Introduction}

\section{Motivation}

The Internet is interconnecting networks all over the world since more than 40 years by 2012. The end-to-end reachability of hosts has always been a basic service of the Internet. However, this reachability is sometimes disrupted for various reasons, such as link or router collapses\citep{}, natural disasters\citep{}, political revolutions\citep{} and human errors\citep{}. There is a real need for methods to systematically detect and locate Internet outages, especially for Internet service providers (ISP). For example are costumers generating support costs for time intensive debugging and outage localization by complaining at the ISP for unreachable networks or the ISP are contractually liable for unreachable networks. An automated, ongoing detection and tracking of connectivity issues of the Internet may generate transparent outage information for customers and enables the ISP to react adequately on a detected reachability problems if possible, for example by changing routes in case of a failure of a transit provider. For this reason, FACT was introduced by \citet{Schatzmann:PAM} as a Flow-based approach of connectivity tracking which is a fully passive approach of identifying missing relies of outgoing flows. The detection of a outage is consolidated by aggregating the unresponsive hosts to network and AS level and rating the severity of the events by affected users.

An obvious caveat of this approach lies in the misinterpretation of service failures to host failures in case no other service is running on the observed host. This problem also exists on higher aggregation level, i.e. if a single host fails the entire network / AS is wrongly detected as down. This is even worse if this service/host are very popular with regard to internal network users.



\section{Related Work}
