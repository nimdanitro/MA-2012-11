

%!TEX root = ./main.tex
%%%%%%%%%%%%%%%%%%%%%%%%%%%%%%%%%%%%%%%%%%%%%%%%%%%%%%%%%%%%%%%%%%%%%%%%%%%%%%%
%%%%%%%%%%%   How to write an abstract  [1]       %%%%%%%%%%%%%%%%%%%%%%%%%%%%%
%%%%%%%%%%%%%%%%%%%%%%%%%%%%%%%%%%%%%%%%%%%%%%%%%%%%%%%%%%%%%%%%%%%%%%%%%%%%%%%
%
% ----------------------------------------------------------
% Goal:
%  1. ... "selling" your work
%  2. ... "selling" your work
%  3. ... "selling" your work
%
% ----------------------------------------------------------
% Checklist:
%
% Motivation:
% - Why do we care about the problem and the results?
% - Importance of your work, the difficulty of the area,
%   and the impact it might have if successful.
%
% Problem Statement:
% - What problem are you trying to solve?
% - What is the scope of your work ?
%
% Approach:
% - How did you go about solving or making progress on the problem?
% - simulation, analytic models, prototype construction?
%
% Results:
% - What's the answer?
% - is so many percent faster, cheaper, smaller, or otherwise better than something else
% - in numbers
% - talk about orders-of-magnitude improvement not small improvements!!!
%
% Conclusions:
% - What are the implications of your answer?
%
% Keywords:
% - ask your supervisor ...
%
%---------------------------------------------------------%
% Limits:
% - Word count limitation: 150 to 200
%
% [1] Philip Koopman, Carnegie Mellon University, 2007
%     How to Write an Abstract
%     http://www.ece.cmu.edu/~koopman/essays/abstract.html
%     10. Sept. 2007
%----------- FORMAT -----------------------------------------------------------
\clearpage \null 
\vfil 
\begin{center}
	\textbf{Abstract} 
\end{center}

Due to commonly occurring losses of the end-to-end reachability of the Internet, 
many different approaches of detecting these outages have been proposed. 
Although, most of them rely on control-plane information or active probing 
measurements, and are thus not suitable for practical usage. In contrast, the 
fully passive approach FACT is able to reliably track remote connectivity 
issues. 
However, network services have different reachability characteristics and are 
partially designed to be temporary unreachable. This different reachability 
characteristics of connection endpoints introduces some noise in the results of 
FACT. By detecting, monitoring and characterizing network services by their 
visibility, popularity and stability, an adjusted FACT mitigates those event 
detection noise by an order of magnitude. Furthermore, the event visibility is 
further improved by accounting only traffic towards these stable network 
services, hence the outage detection is further enhanced. 

\vspace{10em} 
\begin{center}
	\textbf{network outage detection, network service classification} \par 
\end{center}

%----------- FORMAT -----------------------------------------------------------
\vfil

\newpage
\clearpage \null 
\vfil 
\begin{center}
	\textbf{Abriss} 
\end{center}

Die häufig auftrettenden Unterbrüche der Internet End-zu-End-Erreichbarkeit 
führten dazu, dass viele verschiedene Ansätze zur Detektion dieser Probleme 
vorgeschlagen wurden. Jedoch verlassen sich die meisten dieser Ansätze auf 
Informationen der Kontrollebene oder erfordern aktive Messungen, weshalb sie 
nicht geeignet sind für den Praxiseinsatz. Im Gegensatz dazu ermöglicht der 
völlig passive Ansatz FACT eine zuverlässige Überwachung und Detektion von 
entfernten Verbinungsproblemen im Internet. Verschiedene Netzwerkdienste haben  
unterschiedliche Erreichbarkeitscharakteristiken und sind zum Teil dafür 
ausgelegt sind, temporär nicht erreichbar zu sein ohne den Netzwerkbenutzer zu 
beinträchtigen. Dies führt dazu, dass FACT zum Teil Probleme erkennt, welche in 
Wirklichkeit keine sind und als eine Art von Rauschen bezeichnet werden können. 
Durch die Detektion, Überwachung und Charakterisierung der Netzwerkdiense kann 
eine angepasste Version von FACT dieses Rauschen um eine Zehnerpotenz 
verkleinern. Im Weiteren sind die Probleme besser sichtbar 
und die Überwachungmöglichkeiten von FACT erweitert worden.

\vspace{10em} 
\begin{center}
	\textbf{network outage detection, network service classification} \par 
\end{center}

%----------- FORMAT -----------------------------------------------------------
\vfil
