%!TEX root = ./main.tex
\chapter{Of Server Sockets and their Stability}

\section{Server Sockets}
Since the Internet has moved from a research project to a widely used, public communication infrastructure, one of critical success factor was its diversity with respect to network applications or services. This was heavily favored by the Internets layered design as described by the OSI model. 

Todays network applications ranges from traditional services as web, FTP or mail to new and emerging services as video streaming and social networks. However, the term network application or service is overloaded and are differently used depending on the actual technical context.

Since this thesis will operate with flow-level data, layer 5-8 in the OSI model are invisible in the data set. Therefore, network services can be differentiated only by information based on layer 3 and 4 of the OSI model of the two connection end-points. For this reason, the following two abstractions of a connection end-point of a network service are defined:
\begin{defn}{Socket\\}
A socket is uniquely defined by the triple IP address, IP protocol number and protocol port number. A socket is restricted to IP protocol TCP(6) and UDP(17).
\end{defn}

\begin{defn}{Server Socket\\}
	\todo{server socket definition}
\end{defn}

Despite the definition contains the term server, the definition is not only true for server-client application protocols but may also be true for classical P2P-applications, at least for P2P super-nodes. 
% formal definition of socket.. and its function in communication
% formal definition of server socket

\section{Detection of Server Sockets}
% problem of detection with flow-level information (timing issue + flags)
Basically, a server socket can be identified by the fact that a client opens a socket which initiates a connection to a server socket. Usually, the client's socket is chosen at random by his operating system and the server socket should be stable over time since it must offer a specific network service or application. 

Therefore, an straight-forward approach for detecting server sockets is to infer the initiator of the connection by the timing information and determine its opposite as the server socket. However, this approach requires an perfect time synchronization of all netflow exporting devices across the network. In practice, this cannot be achieved in a satisfactory and reliable way.

% connection graph idea... +image
Hence, the detection of server sockets with netflow data relies on the following simple, but beautiful approach. First of all, a communication graph is build. This connection graph consists of nodes each representing an unique socket. If a connection between two sockets is observed, an undirected, unweighted edge between the corresponding two nodes is assigned. This means that the neither the direction nor the weight in terms of packets or bytes are required at all to build the communication graph. 

\missingfigure{Bipartite connection graph}
Since server sockets provide a network application or service, they are likely to be contacted by several clients depending on their popularity. Therefore, it can be assumed that nodes with a high degree correspond to a server sockets.

% introduce minimal vertex cover problem

% greedy algorithm to solve mvcp

% Algorithm of detection



