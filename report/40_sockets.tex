

%!TEX root = ./main.tex
\chapter{Of Server Sockets and their Stability 
\label{chapter:sockets}}

\section{Server Sockets} 
Since the Internet has moved from a research project to
a widely used, public communication infrastructure, one of the critical success
factors was its diversity with respect to network applications or services. This
was heavily favored by the Internets layered design as described by the OSI
model. 

Todays network applications ranges from traditional services as web, FTP or mail
to new and emerging services as video streaming and social networks. However,
the term network application or service is overloaded and are differently used
depending on the actual technical context.

Since this thesis will operate with flow-level data, layer 5-8 in the OSI model
are invisible in the data set. Therefore, network services can be differentiated
only by information based on layer 3 and 4 of the OSI model of the two
connection end-points. For this reason, the following two abstractions of a
connection end-point of a network service are defined:

\parbox{\textwidth}{
	\begin{defn}{\textbf{Socket}\\}
		A socket is uniquely defined by the triple (\textbf{IP address}, 
		\textbf{IP protocol number} and \textbf{protocol port number}). A socket 
		is only defined for IP protocol TCP(6) and UDP(17).
	\end{defn}
}

\parbox{\textwidth}{
	\begin{defn}{\textbf{Server Socket\label{def:serversocket}}\\} 
		A server socket is a socket with a process listening to incoming 
		connections and thus offering a network service. The lifetime of a 
		server socket is not restricted to individual connections, but by the 
		lifetime of the network service.
	\end{defn}
}

\parbox{\textwidth}{
	\begin{defn}{\textbf{Client Socket}\\}
		A client socket is a socket which is only used to initiate a  connection 
		to a server socket. Therefore, client sockets are of temporary lifetime 
		which is limited by the duration this connection.
	\end{defn}
}

In spite of the containment of the term \emph{server} in definition 
\ref{def:serversocket}, this definition is not only valid for server-client 
application protocols, but may also for P2P-applications. 
\todo{Explain more?}

\section{Detection of Server Sockets\label{section:socket_detection}}

% problem of detection with flow-level information (timing issue + flags)
Basically, a \emph{server socket} can be identified by the fact that a client 
opens a socket which initiates a connection to a \emph{server socket}. Usually, 
a \emph{client socket} is chosen at random by his operating system and the 
\emph{server socket} should be stable over time since it must offer a specific 
network service or application. Moreover, on each host a socket can only be assigned to one specific process per instance, i.e. a client socket connection initializing application or a \emph{server sockets} (waiting) network application. Otherwise, a socket-in-use-error is issued by the operating system. 

A straight-forward approach for detecting server sockets is to infer the
initiator of the connection by the timing information and determine its opposite
as the server socket. However, this approach requires a time synchronization of
all netflow exporting devices across the network. In practice, this cannot be
achieved in a satisfactory and reliable way.

% connection graph idea... +image
\todo{Introduce Concentrator} Hence, the detection of server sockets with
netflow data relies on the following approach proposed by
\citet{Schatzmann:Mining,Schatzmann:Dissection, Schatzmann:Tracing}. First of
all, a communication graph is build. This connection graph consists of nodes
each representing an unique socket. If a connection between two sockets is
observed, an undirected, unweighted edge between the corresponding two nodes is
assigned. This means that the neither the direction nor the weight in terms of
packets or bytes are required at all to build the communication graph.

\begin{figure}
	[ht] \centering \missingfigure{bipartite graph}
	
	%\includegraphics[width=\linewidth]{image/bipartite_graph.eps}
	\caption{bipartite graph of cross-boarder connections from a socket to a
	socket} 
	\label{fig:bipartite_graph} 
\end{figure}

Since server sockets provide a network application or service, they are likely
to be contacted by several clients depending on their popularity. Therefore, it
can be assumed that nodes with a high degree correspond to a server sockets.

% introduce minimal vertex cover problem
% greedy algorithm to solve mvcp
% recalling sockets for optimization
% overall Algorithm of detection
\section{Stability Tracking of Server Sockets 
\label{section:socket_tracking}}