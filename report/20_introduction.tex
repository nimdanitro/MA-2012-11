

%!TEX root = ./main.tex
\chapter{Introduction}

% Problem: Connectivity problems exist
The end-to-end connectivity of hosts is the key service of the Internet. However, even after 40 years of intense engineering efforts, this connectivity is temporally broken for various reasons, such as link or hardware failure\citep{}, mis-configurations\citep{}, or natural disasters\citep{}. 

% (Centrality Claim) Why do we care: Requires Troubleshooting Tools (TST) to minimize costs
This shows that there is a real need for methods to systematically detect and locate Internet outages of remote autonomous systems, subnets, and even single hosts. This is particularly true for Internet service providers (ISP), as for example costumers are generating costs for time intensive debugging and support by complaining at the ISP for unreachable networks or the ISP is contractually liable for unreachable networks. An automated, ongoing detection and tracking of connectivity issues of the Internet may generate transparent outage information for customers and enables the ISP to react adequately on a detected reachability problems if possible, for example by changing routes in case of a failure of a transit provider. 

% (What is missing) Introduce the gap that we plan to close: 
Researches and industrial vendors have proposed various approaches for systematically detect, locate and troubleshoot Internet outages and loss of end-to-end reachability.

% State clearly what is missing
However, most of these approaches rely on control-plane information as BGP routing messages or data-plane information achieved by active probing. Both approaches not perfectly suitable for practical usage.

% bash: control plane approaches & active probing approaches
As shown by \citet{Bush:Optometry}, packets in the Internet do not necessarily follow the control plane due to default routes. Moreover, connectivity issues imposed by packet filtering cannot be tracked by control plane approaches \citep{Dainotti:2011:ACI}. Besides legal issues, active probing requires the cumbersome of target selection and increased the load on Internet infrastructure. Furthermore, there is still no active approach which scales well enough for the entire IPv6 address space. Moreover, both approaches are unable to track which part of the Internet is currently actively used by their internal clients. This is required to determine the amount of affected internal clients and therefore to assess the urgency of the outage event. For example, as long as a connectivity issue occurs within an unused remote network, the operator can handle this event with low priority and fix more urgent problems first.

To fill this gap, \citet{SchatzmannPAM2011} proposed a fully passive approach relying on data plane information to identifying remote connectivity problems. The detection of an outage is consolidated by aggregating the unresponsive hosts to network and AS level and rating the severity of the events by affected users. This consolidation is required to reduce the noise of unresponsive hosts caused for example by scanning or botnets and implies an implicit prioritization of the events by the users affected. However, the drawback of this aggregation is that the approach is unaware of service or host outages which may be also important to track, especially if they are important services. Nevertheless, an implicit benefit of this aggregation method is that FACT does not produce any false positive at all.

Nevertheless, FACT's outage detection is solely based on traffic to TCP port 80 in the hope that the process listening to port 80 is a stable service, e.g. a web server. Since TCP traffic to port 80 is sometimes used by various application protocols different than HTTP, e.g. Skype, to traverse firewall and NAT devices, this assumption is not true in general. Depending on the kind of service or application, the characteristics of its stability and uptime differ significantly, i.e. if a host is a web server providing important content which should be up most of the time, or in the case of a Skype super-node, which are temporarily changing over time. However, from a flow perspective the connections to a Skype super-node or to a web server look very similar and are often indistinguishable.

For this reason, the services behind the traffic which FACT is using for outage detection have to be monitored on a longer time scale and are characterized by its stability, relevance and popularity. Afterwards, from these characterized services a smart selection of stable and representative targets is created and fed into FACT for tracking remote connectivity issues. To this end, the observed traffic used for the outage detection can be generalized such that not only traffic to TCP port 80 is considered anymore. 

%The Internet is interconnecting networks all over the world since more than 40 years by 2012. The end-to-end reachability of hosts has always been a basic service of the Internet. However, this reachability is sometimes disrupted for various reasons, such as link or router collapses\citep{}, natural disasters\citep{}, political revolutions\citep{} and human errors\citep{}. There is a real need for methods to systematically detect and locate Internet outages of remote autonomous systems, subnets, and even single hosts if they are of importance. This is particularly true for Internet service providers (ISP), as for example costum·ers are generating costs for time intensive debugging and support by complaining at the ISP for unreachable networks or the ISP is contractually liable for unreachable networks. An automated, ongoing detection and tracking of connectivity issues of the Internet may generate transparent outage information for customers and enables the ISP to react adequately on a detected reachability problems if possible, for example by changing routes in case of a failure of a transit provider. As outlined in section \ref{sec:related_work} in detail, researches and industrial vendors have proposed various approaches for systematically detect, locate and troubleshoot Internet outages and loss of end-to-end reachability. However, most of these approaches rely on control-plane information as BGP routing messages or data-plane information achieved by active probing. For this reason, FACT was introduced by \citet{Schatzmann:PAM} as a flow-based approach of connectivity tracking. This is a fully passive approach for identifying remote connectivity problems and relies solely on flow-level information of cross-border network traffic. This The detection of a outage is consolidated by aggregating the unresponsive hosts to network and AS level and rating the severity of the events by affected users.
%An obvious caveat of this approach lies in the misinterpretation of service failures to host failures in case no other service is running on the observed host. This problem also exists on higher aggregation level, i.e. if a single host fails the entire network / AS is wrongly detected as down. This is even worse if this service or host is very popular with regard to internal network users. Depending on the kind of service, this may be a real problem, i.e. if this host is a web server providing important content, or may be negligible because the service or application connecting to this service is already handling this service outage as in the case of a Skype super-node. However, from a flow-point of view the connections to a Skype super-node or to a web server looks very similar and are often indistinguishable. For this reason, services have to be views at a longer time scale in order to detect services which can be characterized as stable and thus relevant for outage analysis. 

\section{Related Work\label{sec:related_work}}
% Connection Tracking (Active & Passive)
The disruption of Internet end-to-end connectivity is not a new phenomenon, there have been connectivity outages since the early beginning of the Internet. Despite that the end-to-end connectivity is a very basic service of the Internet, the Internet community has not a deeply founded understanding of the problems that causes its disruption \citep{Bush:Optometry}. A dominant share of researchers focussed on "pathological behavior related to the address space, e.g. bogon advertisements \citep{Feamster:2005}, prefix hijacking \citep{Zhang:2010}, BGP misconfigurations \citep{Mahajan:2002} or DDoS attacks \citep{Chen:2001}"\citep{Bush:Optometry}.


% Service Detection (Active & Passive, completeness vs. scalability / importance)


\section{Contribution\label{sec:contribution}}


