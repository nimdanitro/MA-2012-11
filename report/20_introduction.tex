%!TEX root = ./main.tex

\chapter{Introduction}

% Problem: Connectivity problems exist
The end-to-end connectivity of hosts is the key service of the Internet. However, even after 40 years of intense engineering efforts, this connectivity is temporally broken for various reasons, such as link or hardware failure\citep{}, mis-configurations\citep{}, or natural disasters\citep{}. 

% (Centrality Claim) Why do we care: Requires Troubleshooting Tools (TST) to minimize costs
There is a real need for methods to systematically detect and locate Internet outages of remote autonomous systems, subnets, and even single hosts. This is particularly true for Internet service providers (ISP), as for example costumers are generating costs for time intensive debugging and support by complaining at the ISP for unreachable networks or the ISP is contractually liable for unreachable networks. An automated, ongoing detection and tracking of connectivity issues of the Internet may generate transparent outage information for customers and enables the ISP to react adequately on a detected reachability problems if possible, for example by changing routes in case of a failure of a transit provider. 

% (What is missing) Introduce the gap that we plan to close: 
Researches and industrial vendors have proposed various approaches for systematically detect, locate and troubleshoot Internet outages and loss of end-to-end reachability.
% State clearly what is messing
However, most of these approaches rely on control-plane information as BGP routing messages or data-plane information achieved by active probing. 
% bash: control plane approaches & active probing approaches
However, as shown by Bush et al. (cite: Internet Optometry: Assessing the Broken Glasses in Internet Reachability), packets in the Internet do not necessarily follow the control plane. Besides legal issues, active probing requires the cumbersome of target selection and increased the load on Internet infrastructure. Moreover, both approaches lack of the capability to track which part of the Internet is currently actively used by their clients. This is required to determine if a outage event affect at all the client of the users and helps operator to asses the urgency of an event. As example, as long as a connectivity issue occurs within a unused remote network, the operator can this event with low priority and fix urgent problems first.

To fill this gap, Schatzmann et al proposed in \citet{Schatzmann:PAM}
a fully passive approach relying on the data plane information to identifying remote connectivity problems. The detection of a outage is consolidated by aggregating the unresponsive hosts to network and AS level and rating the severity of the events by affected users.


%The Internet is interconnecting networks all over the world since more than 40 years by 2012. The end-to-end reachability of hosts has always been a basic service of the Internet. However, this reachability is sometimes disrupted for various reasons, such as link or router collapses\citep{}, natural disasters\citep{}, political revolutions\citep{} and human errors\citep{}. There is a real need for methods to systematically detect and locate Internet outages of remote autonomous systems, subnets, and even single hosts if they are of importance. This is particularly true for Internet service providers (ISP), as for example costumers are generating costs for time intensive debugging and support by complaining at the ISP for unreachable networks or the ISP is contractually liable for unreachable networks. An automated, ongoing detection and tracking of connectivity issues of the Internet may generate transparent outage information for customers and enables the ISP to react adequately on a detected reachability problems if possible, for example by changing routes in case of a failure of a transit provider. As outlined in section \ref{sec:related_work} in detail, researches and industrial vendors have proposed various approaches for systematically detect, locate and troubleshoot Internet outages and loss of end-to-end reachability. However, most of these approaches rely on control-plane information as BGP routing messages or data-plane information achieved by active probing. For this reason, FACT was introduced by \citet{Schatzmann:PAM} as a flow-based approach of connectivity tracking. This is a fully passive approach for identifying remote connectivity problems and relies solely on flow-level information of cross-border network traffic. This The detection of a outage is consolidated by aggregating the unresponsive hosts to network and AS level and rating the severity of the events by affected users.

An obvious caveat of this approach lies in the misinterpretation of service failures to host failures in case no other service is running on the observed host. This problem also exists on higher aggregation level, i.e. if a single host fails the entire network / AS is wrongly detected as down. This is even worse if this service or host is very popular with regard to internal network users. Depending on the kind of service, this may be a real problem, i.e. if this host is a web server providing important content, or may be negligible because the service or application connecting to this service is already handling this service outage as in the case of a Skype super-node. However, from a flow-point of view the connections to a Skype super-node or to a web server looks very similar and are often indistinguishable. For this reason, services have to be views at a longer time scale in order to detect services which can be characterized as stable and thus relevant for outage analysis. 


\section{Related Work\label{sec:related_work}}


