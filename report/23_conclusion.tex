%!TEX root = ./main.tex
\chapter{Conclusion\label{Conclusion}}

%\section{Conclusion}
Adjusting the traffic preselection of \gls{FACT} from a port-based heuristic to a \gls{server socket} based approach is far more than just a generalization. 

On the one hand, the \glspl{server socket} approach is less affected by scanning and \gls{p2p}, acting as a kind of history based scanning filter. 
Thus, reducing the negative effect of malware and \gls{p2p} churn on the network outage detection. 

On the other hand, a smart composition of the \gls{server socket} set used for the detection process may generally increase the observation coverage of \gls{FACT} without significantly increasing the noise ratio.
Hence, the practical usability of \gls{FACT} is increased even more.

The drawback of the approach is clearly that additional resources are required to store the \glspl{server socket} and the additional effort to detect, monitor and characterize the \glspl{server socket}. 
Nevertheless, the good results from chapter \ref{chapter:results} indicates that server sockets are quite stable over time and that there are even good results possible for events in March, May and August 2010 with sockets detected in 
November 2010. 

However, the selection of \glspl{server socket} can be cumbersome and directly influences the detection sensitivity of \gls{FACT}.
Though, even without any optimization the \gls{server socket} approach is clearly outperforming the original port-based heuristic with respect to the \emph{event-to-noise} ratio. 

\todo{Ausschmücken!! }

\section{Future Work}
% characterize ses by time based activity (night / day, working week, weekend etc.)

In the future, new characterization methods can be investigated, especially the visibility allows to define different metrics, as day / night periods, working week activity or weekend activity.
These new metrics would allow to characterize a \gls{server socket} even more detailed.
Thus, getting a better view on the different properties of these sockets. 

% smarter socket set selection => density of network by choosing only the popularst sockets of each network
Furthermore, the \gls{server socket} set selection can be further enhanced by different techniques.
On the one hand, the density of \glspl{server socket} per prefix/network should be considered and examined.
By deducing the number of sockets required per network so that the probability of seeing traffic towards this network is maximized.
This would be clearly dependent on the single visibility of the selected sockets within this network.
Moreover, the general optimization problem can be mathematically expressed by defining an objective function.
This optimization problem may then be solved by different approaches as simulated annealing or integer linear programming.
In addition, the effect of the number of sockets of a set should be considered. 

% consider different selection methods as clustering analysis approaches, e.g. k-means etc => distance function include also network considerations.. etc. 
On the other hand, different selection approaches can be examined as well.
Given the problem scope of the socket selection based on different characteristics, clustering analysis approaches are worth to be considered. 

% automatization of process => danger of excluding sockets affected by events
% => consider full last week?  
Up to now, the process of detecting, monitoring and characterizing \glspl{server socket} is entirely independent from \gls{FACT}, therefore, an integration of these processing chains into \gls{FACT} should be considered. 
However, there is the danger of not selecting good sockets during potential events affecting these networks.
Hence, good approaches should not considering the (network) event periods during the characterizing phase of \glspl{server socket}.

% near future public release of FlowBox and FACT
Finally, the FlowBox library and the \gls{FACT} source code are intended to be released to public in the very near future, licensed under the Gnu Public License v3.